% Define document class
\documentclass[twocolumn]{aastex631}
\usepackage{showyourwork}

% Begin!
\begin{document}

% Title
\title{Polka-dotted Stars: Mapping the Surface of HAT-P-11}

% Author list
\author{Sabina Sagynbayeva}

% Abstract with filler text
\begin{abstract}
   
\end{abstract}

% Main body with filler text
\section{Introduction}
\label{sec:intro}

\section{The Data}
\section{Hierarchical Bayesian model}
\subsection{The Model}
In this section, we will describe our Gaussian Process (GP) model and the likelihood calculation process used to estimate the model parameters. 

We solve for a large set of parameters that includes the GP hyperparameters, the star's and the planet's orbital parameters, respectively. 
\begin{linenomath}\begin{align}
    \label{eq:largetheta}
    \pmb{\Theta}
     & =
    \left(
    \theta_\bullet
    \,\,\,
    \theta_\star
    \,\,\,
    \theta_p
    \right)^\top
    \quad,
\end{align}\end{linenomath}

Separately, these parameters are defined as 
\begin{linenomath}\begin{align}
    \label{eq:thetastar}
    \pmb{\theta_\star}
     & =
    \left(
    i_\star
    \,\,\,
    m_\star
    \,\,\,
    u_1
    \,\,\,
    u_2
    \,\,\,
    P_\star
    \right)^\top
    \quad,
\end{align}\end{linenomath}
where $i_\star$ is the star's orbital inclination, $m_\star$ is the stellar mass in the units of the solar mass, $u_1$ and $u_2$ are limb-darkening coefficients,
and $P_\star$ is the rotational period of the star.

\begin{linenomath}\begin{align}
    \label{eq:thetap}
    \pmb{\theta_p}
     & =
    \left(
    i_p
    \,\,\,
    e
    \,\,\,
    \psi
    \,\,\,
    \omega
    \,\,\,
    P
    \,\,\,
    t_0
    \,\,\,
    R_p/R_\star
    \right)^\top
    \quad,
\end{align}\end{linenomath}
where $i_p$ is the planet's orbital inclination, $e$ is its eccenticity, $\psi$ is the stellar obliquity, $\omega$ is the argument of pericenter of the planet,
$P$ is the rotational period of the planet, $t_0$ is the transit start time, and $R_p/R_\star$ is the planet to star radius ratio.

We represent the GP hyperparameters as \emph{physically interesting} set of parameters $\pmb{\theta}_\bullet$ \citep{Luger2021}:
%
\begin{linenomath}\begin{align}
        \label{eq:thetaspot}
        \pmb{\theta}_\bullet
         & =
        \left(
        n
        \,\,\,
        c
        \,\,\,
        \mu_\phi
        \,\,\,
        \sigma_\phi
        \,\,\,
        r
        \right)^\top
        \quad,
    \end{align}\end{linenomath}
%
where $n$ is the number of starspots, $c$ is their contrast (defined as the intensity difference between the spot and the 
background intensity, as a fraction of the background intensity),
$\mu_\phi$ and $\sigma_\phi$ are the mode and standard deviation
of the spot latitude distribution, respectively, and $r$ is the radius
of the spots.

We assume that the prior over $y_{true}$ follows a multivariate Gaussian distribution, with a mean vector of zeros and a covariance 
matrix $\Sigma$. We use the quasi-periodic kernel to define the covariance matrix $\Sigma$, which is defined by \texttt{StarryProcess}.
We assume that the observations $y_{obs}$ are corrupted by additive Gaussian noise, such that:
\begin{equation}
    y_{obs} = y_{true} + \epsilon
\end{equation}

where $\epsilon \sim \mathcal{N}(0, \sigma_n^2)$ is the noise term. Given the GP prior and the likelihood function, we can 
calculate the joint posterior distribution over the hyperparameters $\Theta$ and the true function $y_{true}$ given the observed data $y_{obs}$:
\begin{equation}
    P(\Theta, y_{true} \mid y_{obs}) \propto P(\Theta) P(y_{true} \mid \Theta) P(y_{obs} \mid y_{true})
\end{equation}

where $P(\Theta)$ is the prior distribution over the hyperparameters, $P(y_{true} \mid \Theta)$ is the likelihood of the true function given the hyperparameters, and $P(y_{obs} \mid y_{true})$ is the likelihood of the observed data given the true function.
We calculate the log-likelihood function, given by:
\begin{equation}
    \log{P(y_{obs} \mid \Theta)} = -\frac{1}{2} (y_{obs} - \mu)^T \Sigma^{-1} (y_{obs} - \mu) -\frac{1}{2} log |K| - \frac{n}{2} log 2\pi
\end{equation}
or, as defined in eq. 14 of \citep{Luger2021}:
\begin{linenomath}\begin{align}
    \label{eq:log-like}
    \ln \mathcal{L}_m\left(I, P, \mathbf{u}, \pmb{\theta}_\bullet\right)
    =
     & -\frac{1}{2}
    \mathbf{r}_m^\top\left(I, P, \mathbf{u}, \pmb{\theta}_\bullet\right)
    \big[
        \pmb{\Sigma}\left(I, P, \mathbf{u}, \pmb{\theta}_\bullet\right)
        +
        \mathbf{C}_m
        \big]^{-1}
    \mathbf{r}_m\left(I, P, \mathbf{u}, \pmb{\theta}_\bullet\right)
    \nonumber       \\[0.75em]
     & -
    \frac{1}{2}
    \ln \Big|
    \pmb{\Sigma}\left(I, P, \mathbf{u}, \pmb{\theta}_\bullet\right)
    +
    \mathbf{C}_m
    \Big|
    -
    \frac{K}{2}
    \ln \left( 2 \pi \right)
    \quad,
\end{align}\end{linenomath}
where
%
\begin{linenomath}\begin{align}
        \mathbf{r}_m\left(I, P, \mathbf{u}, \pmb{\theta}_\bullet\right)
         & \equiv
        \mathbf{f}_m - \pmb{\mu}\left(I, P, \mathbf{u}, \pmb{\theta}_\bullet\right)
    \end{align}\end{linenomath}
%
is defined as the residual vector,
%
$\mathbf{C}_m$ is the data covariance
(which in most cases is a diagonal matrix whose entries
are the squared uncertainty corresponding to each data point in the light curve),
%
$| \cdots |$ denotes the determinant, and $K$ is the number of data points in
each light curve.%

\section{Simulations}
\subsection{Short light curve}
\subsection{Long light curves (multiple transits)}
\section{Results and Discussion}

\bibliography{bib}

\end{document}
