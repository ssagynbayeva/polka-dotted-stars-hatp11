% Define document class
\documentclass[twocolumn]{aastex631}
\usepackage{showyourwork}
\usepackage{bbold}

% Begin!
\begin{document}

% Title
\title{Polka-dotted Stars: Mapping the Surface of HAT-P-11}

% Author list
\author{Sabina Sagynbayeva}

% Abstract with filler text
\begin{abstract}
   
\end{abstract}

% Main body with filler text
\section{Introduction}
\label{sec:intro}

%
\section{The Data}
The data is collected by the \emph{Kepler} mission and we extracted it using
\texttt{lightkurve}, a Python package for Kepler and TESS data analysis \citep{lightkurve}.
%
%
\section{Hierarchical Bayesian model}

%
\subsection{The Model}
In this section, we will describe our Gaussian Process (GP) model and the likelihood calculation process used to estimate the model parameters. 

We solve for a large set of parameters that includes the GP hyperparameters, the star's and the planet's orbital parameters, respectively. 
\begin{linenomath}\begin{align}
    \label{eq:largetheta}
    \pmb{\Theta}
     & =
    \left(
    \theta_\bullet
    \,\,\,
    \theta_\star
    \,\,\,
    \theta_p
    \right)^\top
    \quad,
\end{align}\end{linenomath}

Separately, these parameters are defined as 
\begin{linenomath}\begin{align}
    \label{eq:thetastar}
    \pmb{\theta_\star}
     & =
    \left(
    i_\star
    \,\,\,
    m_\star
    \,\,\,
    u_1
    \,\,\,
    u_2
    \,\,\,
    P_\star
    \right)^\top
    \quad,
\end{align}\end{linenomath}
where $i_\star$ is the star's orbital inclination, $m_\star$ is the stellar mass in the units of the solar mass, $u_1$ and $u_2$ are limb-darkening coefficients,
and $P_\star$ is the rotational period of the star.

\begin{linenomath}\begin{align}
    \label{eq:thetap}
    \pmb{\theta_p}
     & =
    \left(
    i_p
    \,\,\,
    e
    \,\,\,
    \psi
    \,\,\,
    \omega
    \,\,\,
    P
    \,\,\,
    t_0
    \,\,\,
    R_p/R_\star
    \right)^\top
    \quad,
\end{align}\end{linenomath}
where $i_p$ is the planet's orbital inclination, $e$ is its eccenticity, $\psi$ is the stellar obliquity, $\omega$ is the argument of pericenter of the planet,
$P$ is the rotational period of the planet, $t_0$ is the transit start time, and $R_p/R_\star$ is the planet to star radius ratio.

We represent the GP hyperparameters as \emph{physically interesting} set of parameters $\pmb{\theta}_\bullet$ \citep{Luger2021}:
%
\begin{linenomath}\begin{align}
        \label{eq:thetaspot}
        \pmb{\theta}_\bullet
         & =
        \left(
        n
        \,\,\,
        c
        \,\,\,
        \mu_\phi
        \,\,\,
        \sigma_\phi
        \,\,\,
        r
        \right)^\top
        \quad,
    \end{align}\end{linenomath}
%
where $n$ is the number of starspots, $c$ is their contrast (defined as the intensity difference between the spot and the 
background intensity, as a fraction of the background intensity),
$\mu_\phi$ and $\sigma_\phi$ are the mode and standard deviation
of the spot latitude distribution, respectively, and $r$ is the radius
of the spots.

We assume that the prior over $\mathbb{f}_{true}$ follows a multivariate Gaussian distribution, with a mean vector of zeros and a covariance 
matrix $\pmb{\Sigma}$. We use the quasi-periodic kernel to define the covariance matrix $\pmb{\Sigma}$, which is defined by \texttt{StarryProcess}.
We assume that the observations $\mathbb{f}_{obs}$ are corrupted by additive Gaussian noise, such that:
\begin{equation}
    \mathbb{f}_{obs} = \mathbb{f}_{true} + \epsilon
\end{equation}

where $\epsilon \sim \mathcal{N}(0, \sigma_n^2)$ is the noise term. Given the GP prior and the likelihood function, we can 
calculate the joint posterior distribution over the hyperparameters $\Theta$ and the true function $\mathbb{f}_{true}$ given the observed data $\mathbb{f}_{obs}$:
%
\begin{equation}
    p(\Theta, \mathbb{f}_{true} \mid \mathbb{f}_{obs}) \propto p(\Theta) p(\mathbb{f}_{true} \mid \Theta) p(\mathbb{f}_{obs} \mid \mathbb{f}_{true})
\end{equation}
%
where $P(\Theta)$ is the prior distribution over the hyperparameters, $P(\mathbb{f}_{true} \mid \Theta)$ is the likelihood of the true function given the hyperparameters, and $P(y_{obs} \mid y_{true})$ is the likelihood of the observed data given the true function.
We calculate the log-likelihood function, given by:
%
\begin{linenomath}\begin{align}
    \label{eq:log-likeSabina}
    \ln p(\mathbb{f}_{obs} \mid \pmb{\Theta}) 
    =
    & -\frac{1}{2} (\mathbb{f}_{obs} - \pmb{\mu})^T \pmb{\Sigma}^{-1} (\mathbb{f}_{obs} - \pmb{\mu}) 
    \nonumber       \\[0.75em]
    & -
    \frac{1}{2} \ln |\pmb{\Sigma}| - \frac{n}{2} \ln (2\pi)
    \quad,
\end{align}\end{linenomath}
%
or, as defined in eq. 14 of \citep{Luger2021}:
%
\begin{linenomath}\begin{align}
    \label{eq:log-likeRodrigo}
    \ln \mathcal{L}_m\left(\Theta\right)
    =
     & -\frac{1}{2}
    \mathbf{r}_m^\top\left(\Theta\right)
    \big[
        \pmb{\Sigma}\left(\Theta\right)
        \big]^{-1}
    \mathbf{r}_m\left(\Theta\right)
    \nonumber       \\[0.75em]
     & -
    \frac{1}{2}
    \ln \Big|
    \pmb{\Sigma}\left(\Theta\right)
    \Big|
    -
    \frac{K}{2}
    \ln \left( 2 \pi \right)
    \quad,
\end{align}\end{linenomath}
%
where
%
\begin{linenomath}\begin{align}
        \mathbf{r}_m\left(\pmb{\Theta}\right)
         & \equiv
        \mathbf{f}_m - \pmb{\mu}\left(\pmb{\Theta}\right)
    \end{align}\end{linenomath}
%
is defined as the residual vector,
%
$\Sigma$ is the full covariance, which is defined as 
%
\begin{linenomath}\begin{align}
    \pmb{\Sigma}\left(\Theta\right)
     & \equiv
    \pmb{\Sigma_y} + \pmb{\Sigma_d}
\end{align}\end{linenomath}
%
where $\pmb{\Sigma}(\Theta)$ is the covariance of the distribution over spherical harmonic coefficient
vectors $\mathbb{y}$, and $\pmb{\Sigma_d}$ is the data covariance, which is a diagonal
matrix whose entries are the squared uncertainty $\sigma_m^2$ corresponding to each data point in the light curve.
$| \cdots |$ denotes the determinant, and $K$ is the number of data points in
each light curve.%

\section{Simulations}
\subsection{Short light curve}
\subsection{Long light curves (multiple transits)}
\section{Results and Discussion}

\bibliography{bib}

\end{document}
